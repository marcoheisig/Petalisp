% -*- TeX-master: "petalisp.tex"; TeX-engine: xetex; coding: utf-8 -*-
\documentclass[sigconf]{acmart}
\citestyle{acmauthoryear}

\usepackage{amsthm}
\theoremstyle{definition}
\newtheorem{define}{Definition}

\usepackage{tikz}

% workaround to get Greek letters in typewriter font. Requires XeTeX
\usepackage{fontspec}
\setmonofont{DejaVu Sans Mono}[Scale=0.88]

% fancy boxes around source code examples
\usepackage{alltt}
\usepackage{calc}
\usepackage{fancybox}
\usepackage{lineno}
\setlength{\shadowsize}{1.5pt} \setlength{\fboxsep}{5pt}
\newenvironment{code}
{ \begin{Sbox} \begin{minipage}{\linewidth-26pt}
  \internallinenumbers \begin{alltt} }
{ \end{alltt} \end{minipage} \end{Sbox}
\makebox[\linewidth] { \mbox{\phantom{\tiny 99}}
\shadowbox{\TheSbox} } }

% Fancy display of function signatures
\newenvironment{function}
{\noindent
  \begin{minipage}{\linewidth}
    \setlength\tabcolsep{0pt}
    \begin{tabular}{p{0.86\linewidth}p{0.14\linewidth}}
      \rule{\linewidth}{1pt} & \rule{\linewidth}{1pt} \\
      \rule{0pt}{1.3em} }
{\rule{\linewidth}{1pt} & \rule{\linewidth}{1pt} \\ \end{tabular} \end{minipage} }

\def \N {\mathbb{N}}
\def \Z {\mathbb{Z}}
\def \Q {\mathbb{Q}}
\def \C {\mathbb{C}}

\setcopyright{rightsretained}

% \acmDOI{10.475/123_4}
% \acmISBN{123-4567-24-567/08/06}

\acmConference[ELS 2017]
{10th European Lisp Symposium}
{April 2017}
{VUB - Vrije Universiteit Brussel, Belgium}
\acmYear{2017}
\copyrightyear{2017}

\begin{document}
\title{Petalisp: A language for massively parallel computing}

\author{Marco Heisig}
\affiliation{%
  \institution{FAU Erlangen-N\"urnberg}
  \streetaddress{Cauerstra\ss e 11}
  \city{Erlangen}
  \postcode{91058}
  \country{Germany}
}
\email{marco.heisig@fau.de}

\begin{abstract}
  We present the design and our reference implementation of the parallel
  programming language Petalisp. We discovered that a simple data flow
  language is sufficient to express many practically relevant parallel
  algorithms. This programming model is a possible foundation for powerful
  parallel code generators.
\end{abstract}

\begin{CCSXML}
<ccs2012>
<concept>
<concept_id>10003752.10003753.10003761.10003762</concept_id>
<concept_desc>Theory of computation~Parallel computing models</concept_desc>
<concept_significance>500</concept_significance>
</concept>
<concept>
<concept_id>10011007.10011006.10011008.10011009.10010175</concept_id>
<concept_desc>Software and its engineering~Parallel programming languages</concept_desc>
<concept_significance>500</concept_significance>
</concept>
<concept>
<concept_id>10011007.10011006.10011008.10011009.10010177</concept_id>
<concept_desc>Software and its engineering~Distributed programming languages</concept_desc>
<concept_significance>500</concept_significance>
</concept>
<concept>
<concept_id>10011007.10011006.10011008.10011009.10011012</concept_id>
<concept_desc>Software and its engineering~Functional languages</concept_desc>
<concept_significance>300</concept_significance>
</concept>
<concept>
<concept_id>10011007.10011006.10011008.10011009.10011016</concept_id>
<concept_desc>Software and its engineering~Data flow languages</concept_desc>
<concept_significance>300</concept_significance>
</concept>
<concept>
<concept_id>10011007.10011006.10011041.10011044</concept_id>
<concept_desc>Software and its engineering~Just-in-time compilers</concept_desc>
<concept_significance>300</concept_significance>
</concept>
<concept>
<concept_id>10011007.10011006.10011041.10011048</concept_id>
<concept_desc>Software and its engineering~Runtime environments</concept_desc>
<concept_significance>100</concept_significance>
</concept>
</ccs2012>
\end{CCSXML}

\ccsdesc[500]{Theory of computation~Parallel computing models}
\ccsdesc[500]{Software and its engineering~Parallel programming languages}
\ccsdesc[500]{Software and its engineering~Distributed programming languages}
\ccsdesc[300]{Software and its engineering~Functional languages}
\ccsdesc[300]{Software and its engineering~Data flow languages}
\ccsdesc[300]{Software and its engineering~Just-in-time compilers}
\ccsdesc[100]{Software and its engineering~Runtime environments}

\keywords{High-Performance Computing, Common Lisp}

\maketitle

% -*- TeX-master: "petalisp.tex" -*-
\section{Introduction}
\label{sec:introduction}

The automatic parallelization of computer programs is the holy grail of
parallel computing. The search for this holy grail has spawned many
programming languages, such as High Performance Fortran \cite{HPF}, SISAL
\cite{SISAL}, Fortress \cite{Fortress}, SequenceL \cite{SequenceL}, CM Lisp
\cite{CM-Lisp}, Chapel \cite{Chapel}, Unified Parallel C \cite{UPC} and X10
\cite{X10}. Each of these languages offers a notation for parallel
algorithms and relieves the programmer form the tedious and error-prone
task of communicating data between processes. Yet the majority of parallel
programs in industry and scientific computing are written using classical
general-purpose programming languages. This is not surprising, since no
parallelizing compiler can match or outperform a human expert programmer.

But why is that so? Why are parallelizing compilers consistently worse than
human experts? The key difference is not so much the repertoire of known
optimization techniques, but the understanding of the given program. When a
programmer studies a program, she focuses on the high-level algorithm and
data structures. While doing so, she identifies and skips many irrelevant
sections. She might even find some bugs in the original code. This
illustrates that humans have knowledge far beyond the source code.

In contrast, a machine cannot make assumptions about the meaning of a
program. Without any noteworthy intelligence, it is forced to take the
whole program literally. Consequently, the machine has usually zero
knowledge of the high-level behavior of the program. It only sees a jungle
of data and control flow dependencies. Each of these dependencies can
prevent the whole program from parallelization, even if it belongs only to
a superfluous print statement. In other words, it is impossible to develop
a compiler that reliably parallelizes real world programs. The search for
the holy grail is indeed futile.

This does not mean that computers are bad at generating parallel
programs. The source code of a general purpose program is just a horrible
way to convey meaning to such a code generator. Instead we propose a
specialized programming language to express parallel algorithms at a higher
level. This language must be simple enough that every possible combination
of language primitives is intelligible, yet expressive enough to implement
real world algorithms.

In this paper, we describe the design and implementation of the parallel
programming language Petalisp. In section \ref{sec:examples} we show how it
can be applied to several important parallel algorithms.

A peculiarity of Petalisp is its tight integration into the Common Lisp
programming language. Common Lisp programs are used to create Petalisp
programs, and Petalisp operators take ordinary Common Lisp functions as
arguments\footnote{albeit undefined behavior occurs if these functions have
  side effects}. This approach has two advantages. The first one is that
Petalisp itself is a purely functional language, with all associated
benefits for analysis and optimization. The second one is that Petalisp has
access to every feature and library of Common Lisp. Effectively, Petalisp
is an extension of Common Lisp for massively parallel computing.

% -*- TeX-master: "petalisp.tex"; TeX-engine: xetex; coding: utf-8 -*-
\section{The Design of Petalisp}
\label{sec:design}

If the design goal of Petalisp had to be summarized in one sentence, it
would be:

\begin{quotation}
  Every Petalisp program can be executed efficiently on a parallel
  computer
\end{quotation}

\noindent A consequence of this goal is that other language qualities, such
as expressiveness, are sacrificed where necessary. The design process
thereby turns into a search for the minimal set of features that are
sufficient to denote the majority parallel algorithms. The result of this
minimization problem is a language with only a single data structure and
four primitive operators, which are described hereafter.

\subsection{Strided Arrays}
\label{sec:strided-arrays}

Computation is hardly meaningful without structured data. In classical
Lisps, this role is filled by the \texttt{cons} function, from which all
other data structures can be derived. However, this approach produces data
structures that are far too heterogeneous for any automatic
parallelization. Instead Petalisp uses a data structure that has far more
data regularity: strided arrays.

The viability of array oriented programming has been demonstrated by the
programming language APL \cite{APL}. Strided arrays are an extension of
classical arrays, where the valid indices in each dimension are denoted by
three integers: The smallest admissible index, the step size and the
highest admissible index. More precisely, strided arrays can be defined as:

\begin{define}[strided array]
  A strided array in $n$ dimensions is a function from elements of the
  cartesian product of $n$ ranges to a set of Common Lisp objects.
\end{define}

\begin{define}[range]
  A range with the lower bound $x_L$, the step size $s$ and the upper bound
  $x_U$, with $x_L,\, s,\, u_U \in \Z$, is the set of integers
  $$\{\, x \in \Z \;|\; x_{L} \le x \le x_{U} \;\wedge\; (\exists k \in \Z)\,[ x = x_{L} + k s]\,\}$$
\end{define}

Because strided arrays are so important, they deserve their own
notation. The shape of a strided array is denoted by an S-expression,
starting with the Greek letter σ \,(for \textbf{s}pace or \textbf{s}hape),
followed by lists of integers denoting the lower bound, step size and upper
bound of each range. If the step size is one, it can be omitted. Figure
\ref{fig:sigma-examples} illustrates this convention.

\begin{figure}[h]
  \begin{tabular}{ll}
    \texttt{(σ)}                         & \hspace{-1em} the zero-dimensional space \\
    \texttt{(σ (0 1 8) (0 1 8))}         & \hspace{-1em} index space of a $9 \times 9$ array \\
    \texttt{(σ (0 8) (0 8))}             & \hspace{-1em} ditto \\
    \texttt{(σ (10 2 98))}               & \hspace{-1em} all even two-digit numbers \\
    \texttt{(σ (1 2 3) (1 2 3) (1 2 3))} & \hspace{-1em} corners of a $3 \times 3 \times 3$ cube \\
  \end{tabular}
  \caption{A notation for strided arrays.}
  \label{fig:sigma-examples}
\end{figure}

Strided arrays have several advantages over classical arrays. The lowest
and highest index can be chosen arbitrarily, including from the set of
negative integers \footnote{Coincidentally avoiding any debate over a
  canonical lowest array index.} --- a flexibility that is later used to
define translations of arrays and subsequently to define an unambiguous way
to stack multiple strided arrays next to each other.  The step size
parameter allows to model data that is not contiguous, yet has some level
of regularity.

The original motivation to support array strides stems from the observation
that many parallel algorithms have nearest neighbor data dependencies. To
parallelize them regardless, the domain must be partitioned into multiple
independent sets. This process is called coloring. A simple example of such
a coloring strategy is a chessboard, which ensures that the direct
neighbors of any white field are black, and vice versa. All elements with
the same color can be grouped into a small number of strided arrays. In the
case of the chessboard, two strided arrays are sufficient to describe all
tiles of the same color.

\subsection{Core Operations}

Before the four core operations of Petalisp are discussed, it is
instructive to discuss those features that are not present in
Petalisp. Petalisp lacks any form of control flow --- no conditionals, no
function calls, no jumps and no mechanism for exception handling. Even
more, there is no mechanism for destructive assignment. As a result, a
Petalisp program is nothing more than a pure data flow graph.

The most prominent Petalisp operator is the distributed
\emph{application}. It describes embarrassingly parallel problems, where a
single function is applied to every element of one or more strided arrays,
much like the Lisp function \texttt{map}. A secondary source of parallelism
is introduced by the distributed \emph{reduction} operator, which is
similar to the Lisp function \texttt{reduce}, but where the order of
reduction is unspecified.

\begin{define}[application]
  Let $f$ be a referentially transparent\footnote{A function is
    referentially transparent if it has negligible side effects and same
    arguments always yield the same value.} Common Lisp function that
  accepts $n$ arguments, and let $a_{1}, \ldots, a_{n}$ be strided arrays
  with index space $\Omega$. Then the application of $f$ to
  $a_{1}, \ldots, a_{n}$ is a strided array that maps each index $k$ of
  $\Omega$ to $f( a_{1}(k), \ldots, a_{n}(k) )$.
\end{define}

\begin{define}[reduction]
  \label{def:reduction}
  Let $f$ be a referentially transparent Common Lisp function that accepts
  two arguments, and let $a$ be strided array of dimension $n$, i.e. a
  mapping from each element of the cartesian product of the ranges
  $R_{1}, \ldots, R_{n}$ to some values. Then the reduction of $a$ by $f$
  is a Petalisp data structure of dimension $n-1$ that maps each element
  $k$ of $R_{1} \,\times\, \ldots \, \times \, R_{n-1}$ to the combination of the
  elements $\{ a(i) \,|\, i \in k \,\times\, \Omega_{n} \}$ by $f$ in some
  arbitrary order.
\end{define}

Parallel application and reduction are the only two operators that actually
evaluate Common Lisp functions to compute new values. The remaining two
operators are the \emph{fusion} of several strided arrays, and
\emph{reference} operator, which allow to combine, destructure and reshape
existing strided arrays.

\begin{define}[fusion]
  \label{def:fusion}
  Let $a_{1}, \ldots, a_{n}$ be strided arrays, each mapping from a set of
  indices $\Omega_{k}$ to a set of values.  Furthermore, let the sets
  $\Omega_{1}, \ldots, \Omega_{n}$ be pairwise disjoint. Then the fusion of
  $a_{1},\ldots, a_{n}$ is a strided array that maps each index
  $i \in \bigcup_{k=1}^{n} \Omega_{k}$ to the value of $i$ of the unique data
  structure $a_{k}$ whose domain contains $x$.
\end{define}

\begin{define}[reference]
  \label{def:reference}
  Let $a$ be a strided array with domain $\Omega_a$, let $\Omega_b$ be a
  strided array index space and let $t$ be a transformation from $\Omega_b$
  to $\Omega_a$. Then the reference of $a$ by $\Omega_b$ and $t$ is a
  strided array that maps each $i \in \Omega_b$ to $a(t(i))$.
\end{define}

What has not been specified so far is the space of permissible
transformations in definition \ref{def:reference}, i.e. the possible ways
to reshape an array. This choice is crucial. Limiting the space of
transformations extremely, e.g. only to the identity function, would render
the language more or less useless. A too liberal policy would increase the
complexity of the language to unmanageable levels, conflicting with our
design goal of providing reliable optimization. The next section describes
the compromise that was finally chosen.

\subsection{Permissible Transformations}
\label{sec:transformations}

Study of real world applications has determined, that the following
elementary transformations on strided arrays are particularly useful:

\begin{itemize}
\item \textbf{translation} of indices by a constant, e.g.
  \begin{flushleft}
    $\texttt{(σ (0 9))} \xrightarrow{+\,10} \texttt{(σ (10 19))}$
  \end{flushleft}
\item \textbf{scaling} of indices with a constant, e.g.
  \begin{flushleft}
    $\texttt{(σ (0 4))} \xrightarrow{\times\, 11} \texttt{(σ (0 11 44))}$
  \end{flushleft}
\item \textbf{permuting} the indices of a multidimensional array, e.g.
  \begin{flushleft}
    $\texttt{(σ (-4 4) (2 9))}
    \xrightarrow{1^{\text{st}} \,\leftrightarrow\, 2^{\text{nd}}}
    \texttt{(σ (2 9) (-4 4))}$
  \end{flushleft}
\item \textbf{shrinking} the dimension by dropping some indices, e.g.
  \begin{flushleft}
    $\texttt{(σ (2 2) (3 9))}
    \xrightarrow{\text{drop}\, 1^{\text{st}}}
    \texttt{(σ (3 9))}$
  \end{flushleft}
\item \textbf{increasing} the dimension by adding some indices,e.g.
  \begin{flushleft}
    $\texttt{(σ)}
    \xrightarrow{\text{add}\, 4,\, \text{add}\, 9}
    \texttt{(σ (4 4) (9 9))}$
  \end{flushleft}
\item \textbf{collapsing} one dimension to a single element, e.g.
  \begin{flushleft}
    $\texttt{(σ (2 19) (4 12)}
    \xrightarrow{\text{collapse}\, 1^{\text{st}}\, \text{to}\, 0}
    \texttt{(σ (0 0) (4 12))}$
  \end{flushleft}
\end{itemize}

These six elementary transformations and all possible combinations thereof
form the set of permissible transformations in Petalisp. Understanding and
implementing this set of transformation in Common Lisp is a central
component of this programming model. Transformations are treated as
first-class citizens in Petalisp and have their own notation, which is
similar to the notation of ordinary lambda functions, but starting with the
letter τ and with an implicit values form wrapped around the body, as seen
in figure \ref{fig:transformation-examples}

\begin{figure}[htb]
  \begin{tabular}{ll}
    \texttt{(τ ())} & \hspace*{-1em} the transformation from \texttt{(σ)} to \texttt{(σ)} \\
    \texttt{(τ (i) (+ i 2))} & \hspace*{-1em} shift all indices of a $1$D array by $2$ \\
    \texttt{(τ (m n) m (* n 1/2))} & \hspace*{-1em} scale the second dimension by $1/2$ \\
    \texttt{(τ (m n) n m)} & \hspace*{-1em} invert a matrix \\
    \texttt{(τ (5 a) a)} & \hspace*{-1em} drop the first index if it is 5 \\
    \texttt{(τ (a) a 9)} & \hspace*{-1em} add one dimension with index 9 \\
  \end{tabular}
  \caption{A notation for transformations.}
  \label{fig:transformation-examples}
\end{figure}

\subsection{The Petalisp API}

The core operators of Petalisp --- application, reduction, fusion and
reference --- are simple and orthogonal. They are designed to be easy to
reason about, but not to make it pleasant to write programs with them. This
concern is addressed by the Petalisp API, a set of powerful, smart
functions that expand into one or more invocations of Petalisp core
operators.

\begin{function}
  \texttt{ \textbf{σ} \&rest range-specifications} & \textsl{macro} \\
\end{function}

The macro σ parses the notation for strided arrays from section
\ref{sec:strided-arrays}. Each range specification must be a list of forms,
which are evaluated from left to right in the current environment to
produce the parameters of this range. Returns an object of type
\texttt{strided-array-index-space}.

\begin{function}
  \texttt{ \textbf{σ*} space \&rest range-specifications} & \textsl{macro} \\
\end{function}

Space must be a strided array index space, whose dimension matches the
number of range specifications. Each range specification is parsed as in
the macro σ, but with the variables start, step and end bound to the
respective values of the corresponding range in \texttt{space}. This macro
allows to describe a strided array index space relative to another one,
e.g. stripping the boundary of an existing space. Returns an object of type
\texttt{strided-array-index-space}.

\begin{function}
  \texttt{ \textbf{τ} arguments \&body body} & \textsl{macro} \\
\end{function}

Arguments must be a list of symbols, body must be a list of forms, as shown
in figure \ref{fig:transformation-examples}. The body forms are evaluated
multiple times, with symbols bound to some integers, to determine the
properties of the transformation. Only transformations as specified in
section \ref{sec:transformations} are permitted, an attempt is made to
signal an error in case of invalid transformations. Returns an object of
type \texttt{transformation}.

\begin{function}
  \texttt{ \textbf{α} function \&rest arguments} & \textsl{function} \\
\end{function}

Apply the application operator to the given \texttt{function} and the
\texttt{arguments}. If any of the latter is not a strided array, but a Lisp
scalar or array, it is suitably converted. If, after conversion, the
strided arrays have a different shape, an attempt is made to broadcast them
to a common shape using the reference operator. These references must not
permute, scale or translate their arguments. An error is signaled, when
there is no way to broadcast all strided arrays to a common shape. Returns
an object of type \texttt{strided-array}.

\begin{function}
  \texttt{ \textbf{β} function argument} & \textsl{function} \\
\end{function}

Apply the reduction operator to the given \texttt{function} and
\texttt{argument}. If the latter is a Lisp array, it is converted to a
strided array. Returns an object of type \texttt{strided-array}.

\begin{function}
  \texttt{ \textbf{→} strided-array \&rest modifiers} & \textsl{function} \\
\end{function}

The function → is the generalized reference and broadcast operator. If
\texttt{strided-array} is a Lisp array or scalar, it is suitably converted.
Afterwards, the \texttt{modifiers} are processed from left to right, using
the result of each modification as argument for the next one. Each modifier
must either be a strided array index space or a transformation. If the
current modifier is a transformation, emit a reference node that reshapes
the argument accordingly.  If the current modifier is a strided array index
space and a subspace of the index space of the argument, emit a reference
that selects only this subspace. Otherwise attempt to broadcast the
argument to the given space with a suitable reference. Returns an object of
type \texttt{strided-array}.

\begin{function}
  \texttt{ \textbf{fuse} \&rest arguments} & \textsl{function} \\
\end{function}

All \texttt{arguments} are converted to strided arrays when necessary. The
resulting arrays must must not overlap and are passed to the fusion
operator. Returns an object of type \texttt{strided-array}.

\begin{function}
  \texttt{ \textbf{fuse*} \&rest arguments} & \textsl{function} \\
\end{function}

The fuse* function is similar to the fuse function, but the arguments may
overlap. For indices where some of the arguments overlap, the value of the
rightmost array is chosen. Returns an object of type
\texttt{strided-array}.

% -*- TeX-master: "petalisp.tex"; TeX-engine: xetex; coding: utf-8 -*-
\section{Examples}
\label{sec:examples}

This section contains a collection of Petalisp programs and their
description to familiarize the reader with the previously defined
API.

\subsection{Linear Algebra}

The first batch of examples are from the domain of linear algebra, i.e. the
study of vector spaces and linear mappings between them. Elements and
operations on finite dimensional vector spaces are naturally represented as
vectors and matrices, which are both special cases of strided arrays.

The first example in figure \ref{fig:summation} shows two ways to compute
the sum of six integers, either using an application or a reduction. In
both cases, the functions α and β perform an implicit conversion of their
arguments from lisp objects to strided arrays.

\begin{figure}[h]
\resetlinenumber
\begin{code}
(α #'+ 1 2 3 4 5 6)
(β #'+ #(1 2 3 4 5 6))
\end{code}
\caption{Summation of integers in two different ways.}
\label{fig:summation}
\end{figure}

The dot product of two vectors $\mathbf{a}$ and $\mathbf{b}$ is defined as

\begin{align}
{ \mathbf {a} \cdot \mathbf {b} =\sum _{i=1}^{n}a_{i}b_{i}=a_{1}b_{1}+a_{2}b_{2}+\cdots +a_{n}b_{n}},
\end{align}

i.e. the sum of the products of the corresponding entries in both
vectors. The dot product is widely used, e.g. to compute physical
properties like the mechanical work or the magnetic flux. The corresponding
Petalisp program is shown in figure \ref{fig:dotproduct}.

\begin{figure}[h]
\resetlinenumber
\begin{code}
(β #'+ (α #'* a b))
\end{code}
\caption{The dot product of two vectors a and b.}
\label{fig:dotproduct}
\end{figure}

Norms are an essential tool for many areas of mathematics. One possible
norm for matrices is the row sum norm, which is defined as

\begin{align}
|A||_\infty = \max_{1 \le i \le m} \sum_{j=1}^n |a_{ij}|.
\end{align}

Matrices are the first example of multidimensional strided
arrays. According to definition \ref{def:reduction}, reductions apply to
the last dimension of a strided array, such that a reduction on a
$m \times n$ matrix yields a vector of length $m$ and the reduction of such
a vector yields a scalar.  The program in figure \ref{fig:rowsumnorm}
illustrates this convention, where the innermost reduction computes the sum
of the absolute values in each row, while the outer reduction forms the
maximum over the resulting row sums.

\begin{figure}[h]
\resetlinenumber
\begin{code}
(β #'max (β #'+ (α #'abs A)))
\end{code}
\caption{The row sum norm of a matrix A.}
\label{fig:rowsumnorm}
\end{figure}

The final linear algebra example is the matrix multiplication of two
matrices A and B. It is defined as

\begin{align}
C_{ij} = \sum_{p=1}^{n} A_{ip} B_{pj}.
\end{align}

The corresponding Petalisp program in figure \ref{fig:matmul} is the first
example to use the → function to reshape the matrices A and B. The
$m \times n$ matrix A is transformed to a $m \times 1 \times n$ array and
the $n \times k$ matrix B is transformed to a $1 \times k \times n$
array. Both arrays are then passed to the α function, which detects the
different shape of its arguments, broadcasts both to a common space of size
$m \times k \times n$ and multiplies them element-wise. In other words, $k$
reshaped instances of the matrix A and $m$ reshaped instances of the matrix
B are stacked next to each other before multiplying. In the end the last
dimension of the result of the α function (with size $n$) is summed up,
producing the result with shape $m \times k$.

\begin{figure}[h]
\resetlinenumber
\begin{code}
(β #'+
   (α #'*
      (→ A (τ (m n) (m 1 n)))
      (→ B (τ (n k) (1 k n))))))
\end{code}
\caption{The matrix multiplication of matrices A and B.}
\label{fig:matmul}
\end{figure}

Readers who think the matrix multiplication program in figure
\ref{fig:matmul} is more intricate than a classical version with three
nested loops are reminded that the code in figure \ref{fig:matmul} shows a
\emph{parallel} implementation of a matrix multiplication that may be used
to multiply matrices with a size of several terabytes on a supercomputer.
These four lines of Petalisp are roughly equivalent to parallel
implementations as in ScaLAPACK \cite{slug}, whose matrix multiplication
routine has several hundred lines of code.

\subsection{The Jacobi Method}

Most real world applications on parallel computers are concerned with
numerical algorithms. One such algorithm is the Jacobi method to solve
diagonally dominant systems of linear equations. Such systems arise
frequently in the discretization of partial differential equations
(PDEs). A simple and instructive example of such a PDE is the Laplace
equation with Dirichlet boundary conditions:

\begin{align}
- \Delta u &= 0 \qquad \text{on } \Omega \label{eq:laplace1} \\
 u &= C \qquad \text{on } \partial\Omega. \label{eq:laplace2}
\end{align}

\noindent Equation \ref{eq:laplace1} states that the divergence of the negative
gradient of a given quantity $u$ shall be zero at each point in the domain
$\Omega$. Equation \ref{eq:laplace2} provides the Dirichlet boundary conditions
for the problem, by assigning $u$ fixed values on each point of the domain
boundary $\partial\Omega$. In practice, the Laplace equation occurs e.g. when solving
heat diffusion problems. In this case, $u$ would be the temperature of the
medium. Heat flows from hot regions to colder ones, so the heat flux can be
modeled by the negative gradient of $u$. The conservation of energy implies
that the divergence of the heat flux is zero and leads exactly to the
Laplace equation.

For most domains and boundary values, there is no way to derive an analytic
solution to the Laplace equation. Instead the solution must be approximated
by numerical methods. A classical approach is to replace equation
\ref{eq:laplace1} by a linear system of equations that approximate the
Laplace operator at a finite set of grid points in the interior of $\Omega$. If
a two-dimensional domain is discretized by a rectangular grid with distance
$h$, a linear approximation of the Laplace equation at each point is given
by

\begin{align}
\label{eq:discretized-laplace}
- \Delta u_{x,y} \approx \frac{-4 u_{x,y} + u_{x+h,y} + u_{x-h,y} + u_{x,y+h} + u_{x,y-h}}{h^2} = 0.
\end{align}

\noindent The validity of this formula can be confirmed by Taylor expansion and
solving for the second derivatives. The Jacobi method approximates the
solution of such a system by starting with a random initial guess
$u_{0}$. Based on this initial guess, an improved guess $u_{k+1}$ is
obtained by solving the linear system point-wise, using the values of
$u_{k}$ for each neighbor:

\begin{align}
\label{eq:jacobi-update}
u_{k+1,x,y} = \frac{u_{k,x+h,y} + u_{k,x-h,y} + u_{k,x,y+h} + u_{k,x,y-h}}{4}
\end{align}

This update formula is applied in parallel to the interior point of the
discretized domain $\Omega$. The Dirichlet boundary values are never
modified. The Petalisp implementation of the Jacobi method for this
particular problem is shown in figure \ref{fig:Jacobi}. It starts in lines
2--4 by defining the interior of the given domain with the σ* macro. The
interior is then used in lines 10--13 to select exactly those points of $u$
that are in the interior \emph{after} the specified transformation. Finally
the fuse* function in line 7 determines the next value of $u_{k+1}$ by
overriding the interior of $u_{k}$ according to the update formula from
equation \ref{eq:jacobi-update}.

\begin{figure}[htb]
\resetlinenumber
\begin{code}
(defun jacobi-2d (u iterations)
 (let ((in
        (σ* u ((+ start 1) 1 (- end 1))
              ((+ start 1) 1 (- end 1)))))
   (loop repeat iterations do
     (setf u
       (fuse* u
        (α #'* 0.25
          (α #'+
             (→ u (τ (x y) (1+ x) y) in)
             (→ u (τ (x y) (1- x) y) in)
             (→ u (τ (x y) x (1+ y)) in)
             (→ u (τ (x y) x (1- y)) in))))))
   u))
\end{code}
\caption{The Jacobi Method to solve the equation $- \Delta u = 0$}
\label{fig:Jacobi}
\end{figure}

\subsection{Other examples}

There is a growing number of Petalisp examples on the Petalisp homepage
\footnote{\url{https://github.com/marcoheisig/Petalisp}}, ranging from
simple linear algebra examples to advanced techniques like multigrid
algorithms \cite{briggs2000}. The study of Petalisp examples is currently a
high priority, because we want to make sure that the language is prepared
for the parallel computing needs of the working computer scientist.

% -*- TeX-master: "petalisp.tex" -*-
\section{Implementation}
\label{sec:implementation}


% -*- TeX-master: "petalisp.tex"; TeX-engine: xetex; coding: utf-8 -*-
\section{Conclusions}
\label{sec:conclusions}

Petalisp has been designed as a simple language with only four primitive
operators and a high potential for optimization and
parallelization. Compilers for general purpose programming languages
struggle with compile-time uncertainty and complicated control flow,
whereas in Petalisp, these problems vanish. Its programs are massively
parallel data flow graphs and they are compiled on the fly during the
actual execution.

A high-quality implementation of Petalisp has been written to study this
programming model for real applications. Using this implementation, it has
been demonstrated that parallel HPC algorithms can be written concisely as
Petalisp programs. An unforeseen consequence of the focus on powerful
parallel operations is that it is remarkably pleasant to write Petalisp
programs. Algorithms that would require many nested loops and local
variables in imperative languages can be expressed with only few lines of
Petalisp code. Neither expressiveness, nor elegant notation have been a
primary design goal of Petalisp. The original idea was to find a
fundamental notation for parallel programming, which is well suited for
compilers. It is a relief that the resulting programming language turned
out to be convenient for humans, too.

One final ingredient is still missing before the language can be considered
ready for mainstream adoption: A parallelizing compiler back end. At the
moment, all Petalisp programs are run in serial. The work on this subject
has been postponed while the semantics of the language is still under
development. More algorithms need to be written in Petalisp to determine
whether the current set of features is really sufficient to denote a
reasonable set of parallel algorithms.

Once this work is done, it is possible to proceed to the next chapter of
Petalisp development and investigate parallelization strategies. Any
reasonable parallel scheduler requires detailed knowledge of the given
system resources, so a big portion of this work will be devoted to
performance introspection. The scheduler needs a way to detect the number
of available compute nodes, CPU cores, main memory size, cache hierarchies
and communication channels. Afterwards this information can be combined
with performance models, e.g. Roofline \cite{roofline} or ECM \cite{ecm},
to make qualified scheduling and domain partitioning decisions. The
ambitious long-term goal is that the Petalisp code generator outperforms
even expert HPC programmers.


\section{Acknowledgments}

I am grateful for the support and feedback from Nicolas Neuss, Sebastian
Kuckuk and Harald Köstler. Furthermore I want to thank all maintainers of
Common Lisp packages and developers of Common Lisp implementations. The
excellent quality of the Lisp ecosystem encouraged me to explore Common
Lisp for scientific computing. Finally I want to thank Zachary P. Beane for
his work on Quicklisp and his helpful bug reports.

\bibliographystyle{ACM-Reference-Format}
\bibliography{petalisp}

\end{document}
